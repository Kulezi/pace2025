
\documentclass[a4paper,UKenglish,cleveref, autoref, thm-restate]{lipics-v2021}
%This is a template for producing LIPIcs articles. 
%See lipics-v2021-authors-guidelines.pdf for further information.
%for A4 paper format use option "a4paper", for US-letter use option "letterpaper"
%for british hyphenation rules use option "UKenglish", for american hyphenation rules use option "USenglish"
%for section-numbered lemmas etc., use "numberwithinsect"
%for enabling cleveref support, use "cleveref"
%for enabling autoref support, use "autoref"
%for anonymousing the authors (e.g. for double-blind review), add "anonymous"
%for enabling thm-restate support, use "thm-restate"
%for enabling a two-column layout for the author/affilation part (only applicable for > 6 authors), use "authorcolumns"
%for producing a PDF according the PDF/A standard, add "pdfa"

%\pdfoutput=1 %uncomment to ensure pdflatex processing (mandatatory e.g. to submit to arXiv)
%\hideLIPIcs  %uncomment to remove references to LIPIcs series (logo, DOI, ...), e.g. when preparing a pre-final version to be uploaded to arXiv or another public repository

%\graphicspath{{./graphics/}}%helpful if your graphic files are in another directory

\bibliographystyle{plainurl}% the mandatory bibstyle

\title{PACE Solver Description: DSHunter} %TODO Please add

%\titlerunning{Dummy short title} %TODO optional, please use if title is longer than one line

\author{Paweł Putra}{University of Warsaw, Poland }{pawelputra0@gmail.com}{}{}%TODO mandatory, please use full name; only 1 author per \author macro; first two parameters are mandatory, other parameters can be empty. Please provide at least the name of the affiliation and the country. The full address is optional. Use additional curly braces to indicate the correct name splitting when the last name consists of multiple name parts.


\authorrunning{Paweł Putra} %TODO mandatory. First: Use abbreviated first/middle names. Second (only in severe cases): Use first author plus 'et al.'

\Copyright{Paweł Putra} %TODO mandatory, please use full first names. LIPIcs license is "CC-BY";  http://creativecommons.org/licenses/by/3.0/

\ccsdesc[500]{Theory of computation~Parameterized complexity and exact algorithms} %TODO mandatory: Please choose ACM 2012 classifications from https://dl.acm.org/ccs/ccs_flat.cfm 

\keywords{Dominating Set, PACE 2025, Treewidth, Kernelization, Vertex Cover} %TODO mandatory; please add comma-separated list of keywords

\category{} %optional, e.g. invited paper

\relatedversion{} %optional, e.g. full version hosted on arXiv, HAL, or other respository/website
%\relatedversiondetails[linktext={opt. text shown instead of the URL}, cite=DBLP:books/mk/GrayR93]{Classification (e.g. Full Version, Extended Version, Previous Version}{URL to related version} %linktext and cite are optional

\supplement{The source code of DSHunter is available in \url{https://github.com/Kulezi/pace2025}}%optional, e.g. related research data, source code, ... hosted on a repository like zenodo, figshare, GitHub, ...
%\supplementdetails[linktext={opt. text shown instead of the URL}, cite=DBLP:books/mk/GrayR93, subcategory={Description, Subcategory}, swhid={Software Heritage Identifier}]{General Classification (e.g. Software, Dataset, Model, ...)}{URL to related version} %linktext, cite, and subcategory are optional

%\funding{(Optional) general funding statement \dots}%optional, to capture a funding statement, which applies to all authors. Please enter author specific funding statements as fifth argument of the \author macro.

\acknowledgements{I want to thank prof. Marcin Pilipczuk for guidance and insightful ideas that went into developing this solver.}%optional

\nolinenumbers %uncomment to disable line numbering

% \usepackage{todonotes}

%Editor-only macros:: begin (do not touch as author)%%%%%%%%%%%%%%%%%%%%%%%%%%%%%%%%%%
\EventEditors{John Q. Open and Joan R. Access}
\EventNoEds{2}
\EventLongTitle{42nd Conference on Very Important Topics (CVIT 2016)}
\EventShortTitle{CVIT 2016}
\EventAcronym{CVIT}
\EventYear{2016}
\EventDate{December 24--27, 2016}
\EventLocation{Little Whinging, United Kingdom}
\EventLogo{}
\SeriesVolume{42}
\ArticleNo{23}
%%%%%%%%%%%%%%%%%%%%%%%%%%%%%%%%%%%%%%%%%%%%%%%%%%%%%%

\begin{document}

\maketitle

%TODO mandatory: add short abstract of the document
\begin{abstract}
We describe DSHunter, our submission to the exact Dominating Set track of PACE 2025. DSHunter computes the minimium dominating set of a graph, by preprocessing the graph
into a smaller kernel of an extended Annotated Dominating Set problem instance. 
Kernelized instance is then solved by either a Vertex Cover solver, dynamic programming on a tree decomposition or a Branch\&Reduce algorithm depending on the instances structural properties.
\end{abstract}

\section{Overview}
\label{sec:overview}

DSHunter is an exact solver for finding a minimum dominating set of a graph. DSHunter was developed for the 9th Parameterized Algorithms and Computational Experiments challenge (PACE 2025).
The core idea behind the solver is first preprocessing the given instance into a generalized Annotated Dominating Set (ADS) problem instance.
In some cases the kernel yields an instance that is equivalent to Vertex Cover, and in those cases the solver calls an external \textcolor{red}{Vertex Cover solver WeGotYouCovered~\cite{DBLP:conf/siamcsc/HespeL0S20} from PACE 2019}.
In other cases where the kernel admits small treewidth the solver finds the minimum dominating set directly by dynamic programming on a tree decomposition.
The tree decomposition is found using a heuristic tree decomposition generator  FlowCutter~\cite{DBLP:conf/alenex/HamannS16} from PACE 2017.
In all other cases the solver resorts to a Branch\&Reduce algorithm.


\section{Notation}

Let $G$ be a graph with vertex set $V(G)$ and edge set $E(G)$. The graph $G[X]$ is the induced subgraph of $G$ with vertex set $X$.
The set $N(v)$ denotes the open neighborhood of a vertex $v$ and $N[v] = N(v) \cup \{u\}$ the closed neighborhood of a vertex $v$.
The set $N[X] = \bigcup_{v \in X} N[v]$ denotes the closed neighborhood of a vertex set $X$.
The set $N(X) = N[X] \setminus X$ denotes the open neighborhood of a vertex set $X$.

An Annotated Dominating Set problem instance $I$ is tuple $(G, d, m, f)$, where:
\begin{itemize}
    \item $d: V(G) \to \{\texttt{true}, \texttt{false}\}$, we say that a vertex $v$ is dominated iff $d(v) = \texttt{true}$, such vertices do not need to be dominated by the solution set.
    \item $m: V(G) \to \{\texttt{true}, \texttt{false}\}$, we say that a vertex $v$ is disregarded iff $m(v) = \texttt{true}$, such vertices cannot belong to the solution set.
    \item $f: E(G) \to \{\texttt{true}, \texttt{false}\}$, we say that an edge $e$ is forced iff $f(v) = \texttt{true}$, at least one endpoint of such an edge must belong to the solution set.
    % \item $S$ is a set containing a partial solution containing the vertices of the original Dominating Set problem instance that are already known to belong to the optimal solution set.
    % Such vertices are not present in $G$, i.e. $S \cap V(G) = \emptyset$.
\end{itemize}
That is, a set $X$ is a feasible solution if
\begin{alphaenumerate}
\item every vertex $v$ with $d(v) = \texttt{false}$ is in $N[X]$, 
\item every vertex $x \in X$ satisfies $m(v) = \texttt{false}$, 
\item every edge $uv \in E(G)$ with $f(uv) = \texttt{true}$ satisfies
$\{u,v\} \cap X \neq \emptyset$.
\end{alphaenumerate}
By saying we force an edge $e$ we mean setting $f(e) = \texttt{true}$.

By saying we take a vertex $v$ we mean setting $d(u) = \texttt{true}$ for all $u \in N[v]$, then removing $v$ from $G$, and adding $v$ to the solution set $S$.

By $C(v)$ we denote the set $\{u \in N[v] \mid d(v) = \texttt{false}\}$ or $C(v) = \emptyset$ in the case when $v$ is disregarded, i.e. the set of undominated vertices vertex $v$ can dominate (cover). We call those vertices dominatees of $v$.

By $D(v)$ we denote the set $\{u \in N[v] \mid m(v) = \texttt{false}\}$ or $D(v) = \emptyset$ in the case when $v$ is dominated, i.e. the set of vertices that can dominate $v$. We call those vertices dominators of $v$.

By $N_{exit}(X)$ we denote the set $$\{w \in N(X) \mid N(w) \setminus N[X] \neq \emptyset \vee \exists_{vw \in E(G[N(X)])} f(vw) = \texttt{true}\}$$

By $N_{guard}(X)$ we denote the set $$\{ w \in N(X) \setminus N_{exit}(X)\mid N(w) \cap N_{exit}(X) \neq \emptyset\}$$

By $N_{prison}(X)$ we denote the set $$N[X] \setminus (N_{exit}(X) \cup N_{guard}(X))$$

By $\overline{N_{prison}}(X)$ we denote the set $$N_{prison}(X) \cap \{u \in V(G) \mid d(u) = \texttt{false}\}$$

Note that a Dominating Set problem instance $G$ can be expressed as an Annotated Dominating Set instance $I = (G, d, m, f, \emptyset)$, where $d$, $m$ and $f$ return \texttt{false} for all arguments.

\section{Presolving}
The solver starts by applying several reduction rules in a loop in the following order until all fail to apply. If a reduction rule succeeds, the loop starts from the beginning.
Rules \ref{rule:singledominator} and \ref{rule:samedominators} are based on work by Xiong and Xiao~\cite{DBLP:conf/ijcai/Xiong024}.
Together with the instance the presolver maintains a set $S$ of vertices removed by reduction rules known to belong to the optimal solution.
\subsection{Removal of vertices seeing both ends of a forced edge}
If there exists a vertex $v$ such that there exists a forced edge $vw$, set $d(v) = \texttt{true}$ as it is guaranteed to be dominated by either $v$ or $w$, since one of them must belong to the optimal solution set.
This rule is an exception since it is not a part of the loop, but is applied the moment an edge is being forced, e.g. $f(e)$ changes from \texttt{false} to \texttt{true}.

\subsection{ForceEdgeRule}
If there exists an undominated vertex $u$ of degree 2 with neighbors $v$ and $w$, such that $v$ or $w$ is not disregarded:
\begin{itemize}
    \item if $vw \in E(G)$ and $f(uv) = f(uw) = \texttt{false}$ then remove $u$ from $G$ and force edge $vw$.
    \item if $vw \in E(G)$ and exactly one of the edges $uv$, $uw$ is forced and its non-$u$ endpoint is not disregarded, take this endpoint into $S$.
    \item if $vw \notin E(G)$ and both $v$ and $w$ are dominated, remove $u$ from $G$ from add an edge $vw$ to $G$ and force it.
\end{itemize}

\subsection{Disregard rule}
If there exists $uv \in E(G)$ such that $m(v) = \texttt{false}$, $m(u) = \texttt{false}$, $C(u) \subseteq C(v)$
and there is no forced edge $uw \in E(G)$ s.t $v \neq w$
then disregard vertex $u$, since in an optimal solution we can always replace it with vertex $v$.

\subsection{Remove Disregarded Rule}
If there exists a disregarded, dominated vertex $u$, for all forced edges $(u, v)$ take $v$ and then remove $u$ from $G$. 

\subsection{Single Dominator Rule}
\label{rule:singledominator}
If there exists an undominated vertex $v$ such that $D(v) = \{u\}$, take $u$.

\subsection{Same Dominators Rule}
\label{rule:samedominators}

If there exists a pair of undominated vertices $u$, $v$ such that $D(v) \subseteq D(u)$, mark $u$ as dominated.

\subsection{Alber et al. rules}

We have extended the rules described by Alber et al.~\cite{DBLP:journals/anor/AlberBN06}, to apply to the ADS problem by careful analysis how
they should handle disregarded vertices and forced edges. We did that for \emph{Simple Rules 1, 2, 3 and 4}, as well as for \emph{Main rules 1 and 2} from the referenced work.
\subsubsection{Simple Rule 1}
If there exists an unforced edge that both its endpoints are either dominated or disregarded, remove it from the graph.

If there exists a forced edge $uv \in E(G)$ s.t. $u$ is not disregarded and $v$ is disregarded, take vertex $u$ into the solution set.


\subsubsection{Simple Rule 2}
If there exists an isolated dominated vertex $v$, remove it from the graph.

This rule was also meant apply to dominated vertices of degree 1, however due to an oversight we only apply it to isolated edges, for which it is optimal to take just one of the endpoints.

\subsubsection{Simple Rule 3}
If there exists a dominated vertex $v$ of degree 2 with undominated neighbors $u_1$ and $u_2$, s.t. $u_1 u_2 \in E(G)$ or $u_1$ and $u_2$ share a common, non-disregarded neighbor then:
\begin{itemize}
    \item If the edge $v u_1$ is forced and the edge $v u_2$ is not, take $u_1$.
    \item If both of the edges $v u_1$, $v u_2$ are unforced, remove $v$ from $G$.
\end{itemize}

Note that it is never the case that $u_1 u_2 \in E(G)$ and both $u_1$ and $u_2$ are disregarded as \emph{Simple Rule 1} is applied before \emph{Simple Rule 3}.

\subsubsection{Simple Rule 4}
If there exists a dominated vertex $v$ of degree 3 with undominated neighbors $u_1$, $u_2$ and $u_3$ s.t. $u_1 u_2, u_1 u_3 \in E(G)$ then
\begin{itemize}
    \item If the vertex $u_1$ is not disregarded and the edges $v u_2$, $v u_3$ are unforced, remove $v$ from $G$, also if the edge $v u_1$ was forced take $u_1$.
\end{itemize}


\subsubsection{Main Rule 1}
If there exists a non-disregarded vertex $u$ such that $\overline{N_{prison}}(\{u\}) \neq \emptyset$ then take $u$, and remove $N_{prison}(\{u\})$ and $N_{guard}{(\{u\})}$ from G.

\subsubsection{Main Rule 2}
If there exists a pair of non-disregarded vertices $v$, $w$ such that $\overline{N_{prison}}(\{v, w\}) \neq \emptyset$ and cannot be dominated by a single vertex from $N_{prison}(\{v, w\}) \cup N_{guard}(\{v, w\})$ then
\begin{enumerate}
\item If $\overline{N_{prison}}(\{v, w\})$ can be dominated by both just $v$ and just $w$,
 and there are no forced edges incident to $v$ and $w$, then:


    Let $R = N_{prison}(\{v, w\}) \cup (N_{guard}(\{v, w\}) \cap N(v) \cap N(w))$.
    \begin{itemize}
        \item If $v w \in E(G)$, then force edge $v w$.
        \item If $v w \notin E(G)$ and $|R| > 3$ add three new undominated disregarded vertices $z_1$, $z_2$, $z_3$ and non-forced edges $v z_i$, $w z_i$ for $i \in \{1, 2, 3\}$ to $G$.
    \end{itemize} 
    If one of the above conditions were true, remove the vertex set $R$ from $G$. 
\item If $\overline{N_{prison}}(\{v, w\})$ can be dominated by just $v$ but not $w$, and there are no forced edges incident to $w$ then take $w$ and remove $N_{prison}(\{v, w\})$ and $N_{guard}(\{v, w\}) \cap N(v)$ from $G$.
\item  If $\overline{N_{prison}}(\{v, w\})$ can neither be dominated by just $v$ or just $w$, take both $v$ and $w$ and remove $N_{prison}(\{v, w\}) \cup N_{guard}(\{v, w\})$ from $G$.

\end{enumerate}

\subsection{Local bruteforce rule}
If there exists $X \subseteq V(G)$ and $T_A \subseteq A = N[X] \setminus \{w \in N(X) \mid N (w) \setminus N [X] \neq \emptyset \}$, such that for all possible choices of a proper solution set $T_B \subseteq V[G] \setminus A$ for $I[V(G) \setminus N[X]]$
set $T_A$ is a smallest subset of $A$ such that $T_A \cup T_B$ is a proper solution of $I$,
take all vertices from $T_A$ and disregard all vertices  from $ A \setminus T_A$. 

This rule can be applied in time $\mathcal{O}(|G| \cdot 4^{|A|})$ for a given $X$.

In our solver we limit our choice of sets $X$ to all subsets of the bags of a heurisitcally found tree decomposition with size at most 10, and the sets of all vertices at a distance at most $r$ for $r \in \{0, 1, 2, 3\}$, for which  $|X|, |A| \leq 10$.

\section{Solving the kernel}
After preprocessing is done the solver attempts to solve each of the graph connected components independently using the following methods in order.
\begin{enumerate} 
    \item If the kernelized instance has only forced edges, then it is a Vertex Cover instance, solve it using an external Vertex Cover solver.
    \item If the kernel admits a tree decomposition of a small enough width to fit in the memory limit within 5 minutes of running an external decomposer, solve using dynamic programming on the found decomposition.
    The solver also attempts to reduce the instance by branching on the vertices belonging to the largest bags of the decomposition.
    \item If all previous methods fail, perform an exhaustive Branch\&Reduce search for the optimal solution.
\end{enumerate}

\subsection{Tree decomposition dynamic programming}
Our implementation of tree decomposition dynamic programming for Dominating Set is a direct implementation of the algorithm described in the book Parameterized Algorithms~\cite{DBLP:books/sp/CyganFKLMPPS15}, 
extended to work for ADS. 

The modifications are:
\begin{itemize}
    \item We count the vertex in its \textbf{Forget} node instead of its \textbf{Introduce vertex} node .
    \item A vertex is counted with a weight of $+\infty$ if it is disregarded.
    \item In the \textbf{Introduce Edge}, we return $+\infty$ for the case where no endpoint of this edge is taken into the solution.
\end{itemize}


\subsection{Branch\&Reduce}
Our branching algorithm works by branching first on the vertices with maximum number of incident forced edges. 
If there are no such vertices, the solver branches on the vertex with minimum degree in the graph.
After each decision made by branching the same reductions as in the presolver are applied.
When the graph becomes disconnected as a result of branching the solver solves each component independently using branching.

We use a the size maximal $3$-scattered set of $G$ as a lower bound to discard branches that are guaranteed to produce unoptimal solutions. 

%%
%% Bibliography
%%

%% Please use bibtex, 

\bibliography{solver_description}

\appendix

\end{document}
