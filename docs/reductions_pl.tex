\documentclass[12pt]{article}

% Packages
\usepackage[utf8]{inputenc}
\usepackage[T1]{fontenc}
\usepackage[english]{babel}
\usepackage{amsmath, amssymb}
\usepackage{graphicx}
\usepackage{hyperref}
\usepackage{geometry}
\geometry{margin=1in}

% Title
\title{Różowe wierzchołki}
\author{Paweł Putra}
\date{\today}

\begin{document}

\maketitle

\section{Wstęp}
\emph{Różowy wierzchołek to taki, że zawsze istnieje rozwiązanie optymalne, które go nie zawiera.}

\subsection{Nowe redukcje}

\subsubsection{Redukcja pierwotna}
\emph{Jeśli dla krawędzi $(u, v) \in E$ zachodzi:
\begin{itemize}
\item $N[u] \subseteq N[v]$ 
\item $v$ nie jest różowy
\item nie istnieje \textbf{czerwona} krawędź $(u, w)$ gdzie $w \neq v$ (bo mówi o istnieniu usuniętego sąsiada $u$, który nie jest w $N[v]$)
\end{itemize}}
to pokoloruj $u$ na różowo.

\subsubsection{Redukcja dla pojedyńczego wierzchołka}
\emph{Jeśli istnieje różowy wierzchołek, który jest zdominowany - usuń go z grafu.}

\subsubsection{Redukcja sąsiedztwa}
\emph{Jeśli istnieje \textbf{niezdominowany} nieróżowy wierzchołek $v$, że całe $N(v)$ jest różowe, weź go do rozwiązania i usuń $N[v]$ z grafu.}

\subsection{Wpływ na AlberSimpleRule1}
Było: \emph{Jeśli istnieje \textbf{czarna} krawędź, której oba końce są zdominowane - usuń ją.}\\

Teraz z różowymi wierzchołkami mamy dodatkowo:
\begin{itemize}
\item \emph{Jeśli istnieje \textbf{czarna} krawędź, której oba końce są różowe - usuń ją.}\\
\item \emph{Jeśli istnieje \textbf{czerwona} krawędź $(u, v)$, taka, że $u$ jest różowy, dodaj $v$ do rozwiązania i usuń wierzchołki $u$ i $v$.}
\item \textbf{Test:} \emph{Jeśli istnieje \textbf{czerwona} krawędź, której oba końce są różowe, mamy sprzeczność - instancja jest niepoprawna.} 

\end{itemize}


\subsection{Wpływ na AlberSimpleRule\{2, 3, 4\}}
Zauważmy, że usuwany w redukcji wierzchołek $u$ nie może być różowy, bo musi być zdominowany, a więc nie może go być w grafie po redukcji z $(1.1.2)$.
Zostaje więc spojrzeć co się dzieje jak w sąsiedztwie są różowe wierzchołki.

\subsubsection{AlberSimpleRule2 (usuwanie zdominowanego liścia)}
Bez zmian, różowy sąsiad nie przeszkadza usunąć wierzchołka $u$.

\subsubsection{AlberSimpleRule3.1}
Było: \emph{Jeśli istnieje zdominowany wierzchołek $u$ o stopniu 2, z niezdominowanymi sąsiadami $v$ oraz $w$ połączonymi krawędzią, usuń $u$ z grafu. }

Można było tak zrobić, bo mieliśmy gwarancję, że jeśli istnieje optymalny zbiór dominujący zawierający wierzchołek $u$ to można go podmienić na któregoś z sąsiadów bez utraty optymalności. 

Jeśli wierzchołki $v$ i $w$ są różowe, to nie mamy takiej gwarancji, w szczególności gdy $V = \{u, v, w\}$ to jedyny zbiór dominujący to $\{u\}$.

Zatem nowa reguła brzmi:

\emph{Jeśli istnieje zdominowany wierzchołek $u$ o stopniu 2, z niezdominowanymi sąsiadami $v$ oraz $w$ połączonymi krawędzią, 
\textbf{z których conajwyżej jeden jest różowy}, usuń $u$ z grafu. }

\subsubsection{AlberSimpleRule3.2}
Było: \emph{Jeśli istnieje zdominowany wierzchołek $u$ o stopniu 2, z niezdominowanymi sąsiadami $v_1$ oraz $v_2$ ze wspólnym sąsiadem $w$ $(w \neq u)$, usuń $u$ z grafu. }

Można było tak zrobić, bo mieliśmy gwarancję, że jeśli istnieje optymalny zbiór dominujący zawierający wierzchołek $u$ to można go podmienić na $w$ bez utraty optymalności. 

Jeśli wierzchołek $w$ jest różowy, to nie mamy takiej gwarancji.

Zatem nowa reguła brzmi:

\emph{Jeśli istnieje zdominowany wierzchołek $u$ o stopniu 2, z niezdominowanymi sąsiadami $v_1$ oraz $v_2$ ze wspólnym \textbf{nieróżowym} sąsiadem $w$ $(w \neq u)$, usuń $u$ z grafu. }


\subsubsection{AlberSimpleRule4}
Było: \emph{Jeśli istnieje zdominowany wierzchołek $u$ o stopniu 3, z niezdominowanymi sąsiadami $v_1$, $v_2$ oraz $v_3$, takimi, że istnieją krawędzie $(v_1, v_2)$ i $(v_2, v_3)$ usuń $u$ z grafu. }

Można było tak zrobić, bo mieliśmy gwarancję, że jeśli istnieje optymalny zbiór dominujący zawierający wierzchołek $u$ to można go podmienić na $v_2$ bez utraty optymalności. 

Jeśli wierzchołek $v_2$ jest różowy, to nie mamy takiej gwarancji.

Zatem nowa reguła brzmi:

\emph{Jeśli istnieje zdominowany wierzchołek $u$ o stopniu 3, z niezdominowanymi sąsiadami $v_1$, $v_2$ oraz $v_3$, takimi, że istnieją krawędzie $(v_1, v_2)$ i $(v_2, v_3)$, \textbf{oraz $v_2$ nie jest różowy}, usuń $u$ z grafu. }

\subsection{Wpływ na ForcedEdgeRule}
Czerwona krawędź $(u, v)$ oznacza, że w optymalnym rozwiązaniu $S$ musi być conajmniej jeden element zbioru $\{u, v\}$.

Oznaczając krawędź $(u, v)$ jako czerwoną oznaczamy również $N[u] \cap N[v]$ jako zdominowane, bo będą zdominowane przez $u$ lub $v$.

\subsubsection{Wierzchołki o stopniu 2}
\emph{Jeśli istnieje \textbf{czarny} wierzchołek u o stopniu 2, taki, że jego sąsiedzi v, w są połączeni krawędzią, to:
\begin{itemize}
    \item \textbf{Test:} Zakładając aplikowanie redukcji $(1.1.3)$ przed tą, nie istnieje sytuacja, w której zarówno wierzchołek v jak i w jest różowy. 
    \item jeśli obie krawędzie wierzchołka u są czarne, oznacz krawędź (v, w) jako czerwoną i usuń wierchołek u.
    \item \textbf{Test:} Zakładając aplikowanie redukcji $(1.2)$ przed tą, nie istnieje sytuacja, w której jakiś koniec czerwonej krawędzi jest różowy.
    \item jeśli tylko krawędź (u, w) jest czerwona, weź w do rozwiązania i usuń wierzchołek u.
    \item jeśli tylko krawędź (u, v) jest czerwona, weź v do rozwiązania i usuń wierzchołek u.
    \item jeśli obie krawędzie są czerwone - \textbf{nie rób nic} (pomysł: ściągnąć do jednego niezdominowanego wierzchołka $g$, wtedy $g \in S' \implies \{v, w\} \subseteq S$,
    a $g \not\in S' \implies u \in S$).
\end{itemize}
}

Czyli pod założeniem wyczerpania poprzednich redukcji bez zmian.

\subsection{Wpływ na AlberMainRule1}
Ta redukcja polega na zachłannym wzięciu wierzchołka $v$ jeśli jego sąsiedztwo spełnia odpowiednie kryteria.

\textbf{Dodatek:} \emph{Rozważany wierzchołek nie może być różowy.}

\subsection{Wpływ na AlberMainRule2}
Ta redukcja polega na zachłannym wzięciu wierzchołka $v$ i/lub $w$ jeśli ich sąsiedztwo spełnia odpowiednie kryteria.

\textbf{Dodatek:} \emph{Rozważane wierzchołki nie mogą być różowe.}


\end{document}